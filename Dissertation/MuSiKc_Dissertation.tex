\documentclass[a4paper]{amsart}
\usepackage{amsmath, amsthm}
\usepackage{amssymb}
\usepackage{amsfonts}

\newtheorem{thm}{Theorem}[section]
\newtheorem{lem}[thm]{Lemma}
\newtheorem{prop}[thm]{Proposition}
\newtheorem{cor}[thm]{Corollary}
\newtheorem{defn}{Definition}[section]
\newtheorem{example}{Example}[section]
\newtheorem{exercise}{Exercise}[section]
\newtheorem{algorithm}{Algorithm}[section]


\newcommand\be{\begin{equation}}
\newcommand\ee{\end{equation}}
\newcommand\half{\frac{1}{2}}

\newcommand\CC{{\mathbb C}}
\newcommand\EE{{\mathbb E}}
\newcommand\NN{{\mathbb N}}
\newcommand\OO{{\cal O}}
\newcommand\PP{{\mathbb P}}
\newcommand\PPbar{{\overline{\mathbb P}}}
\newcommand\RR{{\mathbb R}}
\newcommand\TT{{\mathbb T}}
\newcommand\ZZ{{\mathbb Z}}

\renewcommand{\theenumi}{(\roman{enumi})}    

\DeclareMathOperator{\conv}{co}
\DeclareMathOperator{\GL}{GL}
\DeclareMathOperator{\Span}{Span}
\DeclareMathOperator{\Vol}{Vol}

\newcommand\fhat{{\widehat{f}}}
\newcommand\ghat{{\widehat{g}}}
\newcommand\Lhat{{\widehat{L}}}
\newcommand\LAhat{{\widehat{L^A}}}
\newcommand\Lhhat{{\widehat{L^h}}}
\newcommand\phihat{{\widehat{\phi}}}
\newcommand\phiAhat{{\widehat{\phi^A}}}
\newcommand\phihhat{{\widehat{\phi^h}}}


\newcommand{\va}{{\bf a}}
\newcommand{\vb}{{\bf b}}
\newcommand{\vc}{{\bf c}}
\newcommand{\vh}{{\bf h}}
\newcommand{\ve}{{\bf e}}
\newcommand{\vg}{{\bf g}}
\newcommand{\vp}{{\bf p}}
\newcommand{\vq}{{\bf q}}
\newcommand{\vu}{{\bf u}}
\newcommand{\vv}{{\bf v}}
\newcommand{\vw}{{\bf w}}
\newcommand{\vx}{{\bf x}}
\newcommand{\vy}{{\bf y}}
\newcommand{\vz}{{\bf z}}

\newcommand{\vL}{{\bf L}}
\newcommand{\vT}{{\bf T}}

\newcommand{\Balpha}{{\mbox{\boldmath $\alpha$}}}
\newcommand{\Bbeta}{{\mbox{\boldmath $\beta$}}}
\newcommand{\Bgamma}{{\mbox{\boldmath $\gamma$}}}
\newcommand{\Bell}{{\mbox{\boldmath $\ell$}}}
\newcommand{\Blambda}{{\mbox{\boldmath $\lambda$}}}
\newcommand{\Bmu}{{\mbox{\boldmath $\mu$}}}

\usepackage{graphicx}
\usepackage{physics}
\usepackage{amsmath}
\graphicspath{ {images/} }

\begin{document}

\title[Multi-Level Sparse Grid Interpolation Kernel Collocation (MuSiKc)]
{The Multi-Level Sparse Grid Interpolation Kernel Collocation (MuSiK-c) Algorithm, Applied to Basket Options}
\author[NicholasWilton]{Nicholas Wilton}


\maketitle

\section{Introduction}

Numerical Analysis is the study and use of computation algorithms in approximation within the field of mathematical analysis of which (and in particular interest to this paper), partial differential equations (PDEs) are common throughout many areas of science. Whilst analytic solutions exist for some PDEs using such techniques as 'seperation of variables', 'integral transform' and 'change of variables', often either no solution exists orthe analytic tools are currently insufficient to find it.   

The known algorithms developed for solving PDEs are generally classified into nine main methods:\\
\\
\textperiodcentered Finite Difference\\
Where functions whose known values at certain points are approximated by differences between these values. The grid on which these points (aka. nodes) reside is often referred to as a mesh there each node in a finite difference scheme is connected to it's neighbour in either a forward or backwards manner by the algorithm employed\\
\\ 
\textperiodcentered Method of Lines\\
PDE's in multiple dimensions are discretised into Ordinary Differential Equations (ODEs), in all but one dimension allowing for any of the vast number of numerical integration solvers to be used in these dimensions, meanwhile the final dimension is solved by ...\\
\\ 
\textperiodcentered Finite Element\\
Used for approximating boundary value problems where by the solution is approximated by the composition of many linear elements again using a mesh of nodes connected to each other.\\
\\
\textperiodcentered Gradient Discretisation\\


\textperiodcentered Finite Volume
Similar to both Finite Element and Finite Difference methods, a mesh is created on which values are calculcated at discrete points. Using the diveregence theorem to convert volume integrals into surface integrals around each point...\\
\\ 
\textperiodcentered Spectral\\
Where approximation is done by combining a series of basis functions by superposition and then choosing the co-efficients of the series which minimise the error of the result. For example a Fourier series of sinusoidal waveforms or Radial Basis Functions (RBF).\\
\\ 
\textperiodcentered Meshfree\\
In contrast to the previous methods, which all require connections between nodes of a grid a mesh free method requires no connection between nodes.
\\ 
\textperiodcentered Domain Decomposition\\
Whereby a boundary value problem is split into smaller problems on sub-domains each of which is independent from the others allowing for parallelisation of the overall global problem.
\\ 
\textperiodcentered Multigrid\\
Using a heirachy of descritised grids with different levels of coarseness between the nodes. The idea being that the convergence of an iterative method can be accelerated by solving on a coarser grid to make a global correction to the finer grid.
\\
Recent work by

\section{Theoretical Background}
An in-depth discussion of the theoretical background to this work can be found in \cite{mski0}, whilst here we present a more brief overview of the main points.

\section{Partial Differential Equations}

\subsection{Elliptical PDEs}
If we define a linear operator $L : C^{2} (\Omega) \rightarrow C(\Omega)$ as an elliptic differrential operator on u(x) as:
\be
Lu(x) = \sum_{i,j=1} a_{i,j}(x) \pdv{}{x_i}{x_j} u(x) + \sum_{i=1} b_{i}(x) \pdv{}{x_i} u(x) + b_{0}(x) u(x)
\label{EPDE}
\ee
where the coefficient matrix $\big[ a_{ij}(x) \big] \in \Re^{dxd}$ satisfies the condition:
\begin{align*}
\exists \mbox{  } \alpha > 0
\end{align*}
 such that,
\begin{align*}
\sum_{i,j=1}^{d} a_{ij}(x)c_{i}c_{j}\geq \alpha||c||_{2}^{2}
\end{align*}

for all $ x \in \Omega and c \in \Re^{d}$

For a boundary value problem, we would then solve the second order elliptic PDE with the boundary conditions:

\be
Lu = f \mbox{ in } \Omega \\
\label{EPDE1}
\ee
\be
u = g \mbox{ on } \partial \Omega
\label{EPDE2}
\ee

Where again, L is the elliptic operator and f and g are the functions describing the boundary.

\subsection{Parabolic PDEs}


\be
Lu(x,t) = u_{t} \sum_{i,j=1} a_{i,j}(x,t) \pdv{}{x_i}{x_j} u(x,t) - \sum_{i=1} b_{i}(x,t) \pdv{}{x_i} u(x,t) - c(x,t) u(x)
\label{PPDE}
\ee

The Black-Sholes equation is a 2nd order parabolic PDE of the form

\subsection{Radial Basis Functions}

A radial basis function (RBF) is a real valued function whose value only depends on the ditance from the origin or centre such that:

\be
\phi(x,c) = \phi(||x-c||)
\ee

Where x is the point of interest and $x_{i}$ is the location of the central point. \\Some examples of RBFs are the Euclidean distance:

\be
\phi(r) = \sqrt{x^{2} + y^{2}}
\ee
Hardy's Multiquadric RBF:
\be
\phi(r) = \sqrt{c^{2} + ||x-x_{i}||^{2}}
\ee

The Gaussian RBF
\be
\phi(r) = e^{-||x-x_{i}||^{2} / c^{2}}
\ee


Of particular interest are the Multiquadric (MQ) and Gaussian RBFs as they are both inifintely differentiable and have been shown to exhibit accuracy, stability and ease of implementation in for example Franke (1982)\cite{rbf3}. As such they have become popular in the literature of various interpolation schemes \cite{rbf4}. In both cases a parameter \em{c} is defined, known as the \em{shape parameter} the size of which will sharpen (decreasing c) or flatten (increasing c) the function.
[TODO insert some graphs]

It has been shown that a larger value of c will increase accuracy but exceed a limit and the system will become ill conditioned and unstable. Likewise reduceing c will improve the conditioning but also lead to an inaccurate solution. There has been significant effort devoted to finding the optimal value of c for different RBFs for example \cite{} however this is still considered an open question within the field.

Ansiotropic Radial Basis Functions
If the domain of interest is not the same size in all dimensions then an RBF becomes ansiotropic. To model this let $\phi(||\dot -x_{i})$ is some RBF centred around $x_{i} \in \Re^{d}$ and $A \in \Re^{d x d}$ is an invertible transformation matrix, then the ansiotropic radial basis function $\phi_{a}$ is defined by:
\be
\phi_{A}(||\cdot-x_{i}||)=\phi(||A(\cdot-x_{i}||)
\ee
Furthermore we can then define the ansiotropic tensor based product function (ATBPF) of the MQ and Gasussian basis functions respectively, as:

\be
\phi_{A,x_{i}}(x)= \prod^{d}_{k=1}\sqrt{A^{2}_{k}(x_{k}-x^{k}_{i})^{2}+c^{2}_k}
\ee

\be
\phi_{A,x_{i}}(x)= \prod^{d}_{k=1}\exp{\frac{-A^{2}_{k}(x_{k}-x^{k}_{i})^{2}}{ c^{2}_k}}
\ee
Where, k is the $k^{th}$ dimension of x and $A_{k}$ is $k^{th}$ diagonal element of $A \in \Re^{d} x \Re^{d}$.

Whilst we can observe that the Gaussian ATBPF still belongs to the family of RBFs, the MQ ATPBF is no longer radially symmetric.

Now, if we let $Ch_{k} = c_{k} / A_{k}$ we find:

\be
\phi_{A,x_{i}}(x)= \prod^{d}_{k=1}\sqrt{(x_{k}-x^{k}_{i})^{2}+Ch^{2}_k}
\ee

\be
\phi_{A,x_{i}}(x)= \prod^{d}_{k=1}\exp{\frac{(x_{k}-x^{k}_{i})^{2}}{ Ch^{2}_k}}
\ee

that $Ch_{k}$ represents a 'shape parameter' as noted previously where $h_{k}$ represents the distance between nodes in the kth direction and $A_{k}$ is the number of nodes in that same direction minus one.

The first and 2nd derivtives of the ATPBFs are then in the Mutliquadric case,

\be
D_{x_{p}}(\phi_{A,x_{i}})=\frac{x_{p}-x^{p}_{i}}{\sqrt{(x_{p}-x^{p}_{i})^{2} + (Ch_{p})^2 }} \prod^{d}_{k\neq p}\sqrt{(x_{k}-x^{k}_{i})^{2}+Ch^{2}_k}
\ee
\be
D_{x_{p}}^{2}\phi_{A,x_{i}}(x)=\frac{(Ch_{p})^{2}}{[(x_{p}-x^{p}_{i})^{2} + (Ch_{p})^2 ]^{3/2}} \prod^{d}_{k\neq p}\sqrt{(x_{k}-x^{k}_{i})^{2}+Ch^{2}_k}
\ee
and in the Gaussian case
\be
\phi_{A,x_{i}}(x)= \prod^{d}_{k=1}\exp{\frac{(x_{k}-x^{k}_{i})^{2}}{ Ch^{2}_k}}
\ee


Sums of RBFs can be very useful in approximating functions in a similar way to summing sinusoidal functions using Fourier Series expansion leads to the approximation of a periodic function.\\

\subsection{Spectral Methods}

Describe issues with Gibbs phenomena


\subsection{Kansa method}

Kansa's method \cite{rbf1} \cite{rbf2} is a spectral method of approximating u(x) via:

\be
u(x) = \sum^{N}_{i=1} \lambda_{i} \Phi(||x - x_{i})
\label{RBFA}
\ee

Where $\Phi$ is the radial basis function of choice.

For Kansa's method we choose a $\Xi = \Xi_{1} \cup \Xi_{2}$ which we will call central nodes and where $\Xi_{1} \in \Omega$ (i.e are interior points) whilst  $\Xi_{2} \in \partial \Omega$ exist on the boundary $\partial \Omega$

The key of course, is to find the $\lambda$ coefficients for each of the summation terms that make the best approximation of the function u(x). For the elliptical PDE of (\ref{EPDE}) we can substitute (\ref{RBFA}) into the boundary counditions (\ref{EPDE1}) and (\ref{EPDE2}) to get:

\be
\sum_{i=1}^{N} \lambda_{i} L \phi_{x_{i}}(x_{j}) = f(x_{j}),\mbox{for j = 1,2,...n}
\ee

\be
\sum_{i=1}^{N} \lambda_{i} \phi_{x_{i}}(x_{j}) = g(x_{j}),\mbox{for j = n+1,n+2,...N}
\ee

Whilst much work using collocation with RBFs had been performed using Kansa's method to solve elliptic boundary value problems, it wasn't until Myers et al. \cite:{rbf0} proposed the space time method that applications for parabolic problems were first successfully investigated.


\subsection{Space-time method}

In this chapter, we firstly review one well-known collocation method called the Kansa method which is utilised in this thesis in Section 3.1. We then introduce two main methods used for
the parabolic problem (see Definition 3.2), the Method of Lines (MOL) and the space-time method in Section 3.2. In Section 3.3, we present one option pricing example to show the performance of the space-time method and the MOL when solving a parabolic problem.



\subsection{Finite Difference}

\subsection{Method of Lines}
Describe how combinationation technique of MoL reduces Gibbs problem

\subsection{Sparse Grid Collocation}

In the approximation field, high dimensional problems are always difficult because
of the curse of dimensionality. Floater and Iske [36] proposed a multilevel interpolation
scheme to circumvent this problem. The multilevel interpolation method requires
decomposing the given data into a hierarchy of nested subsets. In [68, 69], Iske
further studied the scheme and gave an efficient construction of such hierarchies.
In [70], Iske and Levesley developed the multilevel scheme based on adaptive
domain decomposition. Based on Floater-Iske setting, Narcowich, Schaback and
Ward [91] demonstrated the multilevel method is a numerically stable method for
the interpolation and gave some theoretical underpinnings. Further, Hales and
Levesley [55] demonstrated the error estimates for the multilevel approximation
using polyharmonic splines. Fasshauer and Jerome used the multilevel method
with compactly supported radial basis functions (CSRBFs) to solve elliptic PDE
in [28, 30]. In [26], Farrell and Wendland also used the multilevel RBF collocation
method with CSRBFs to solve elliptic PDEs on bounded domains. Moreover, they
demonstrated a convergence theory.
Another way to overcome the problem is the sparse grid method introduced by
Zenger [126]. This method relies on a multi-scale basis via a tensor product
construction and saves a massive amount of storage and memory cost without
45
Multilevel sparse grid kernel collocation with RBFs 46
loosing accuracy. Hemker [58] applied the finite volume method on sparse grids to
solve three-dimensional elliptic problems. In [54], Griebel, Schneider and Zenger
developed a combination technique for the sparse grid. They also demonstrated
that the combination approach works for both smooth solutions and non-smooth
solutions of linear problems, and even for non-linear problems. Griebel [52] employed
finite difference in multilevel sparse grid method to solve elliptic PDEs. In 2013,
Georgoulis, Levesley and Subhan [48] proposed an method called multilevel sparse
grid kernel (MuSIK) for interpolation. Here, we extend this MuSIK method to
the collocation problem.

One of the advantages in using radial basis function is easy to construct even in
high-dimensional problems. However, in order to achieve accuracy when dimension
d is increasing, we have to fix the fill distance of full grid. That means the number
of evenly distributed collocation points in every direction N is constant. As a
result, the size of a full grid is growing exponentially as Nd. In contrast, the sparse
grid kernel (SIK) algorithm which combines approximations based tensor product
anisotropic radial basis functions on every sub-grid is a stable and efficient method
when facing high dimension problem. The support of this matter is that under
the assumption of sufficient smoothness of the data, the amount of nodes utilised
can be reduced dramatically to guarantee a certain accuracy based on carefully
constructed tensor product anisotropic basis function. Owing to the additional
smoothness assumed, there is only a negligible loss of precision. The basic idea of
SIK was first introduced about fifty years ago in [1, 106] and Zenger [126] proposed
sparse grid methods in 1991.

 Schreiber discussed tensor product of one-dimensional RBFs applying directly
sparse grid methods in her thesis [103], where numerical results corresponding to 
the resulting method were not promising. On the other hand, the direct using
of non-tensor product RBFs in the sparse grid setting is not straightforward,
since the approximation spaces are characterised by basis functions with different
anisotropic scaling in various directions. By utilising such scaling, the solution
obtained from sparse grid method is infeasible as there is no guarantee about the
well-posedness of the resulting kernel-based interpolation problems.
The strategy we adopt here is a sparse grid combination technique which was
introduced in [54], afterwards this technique is operative in piecewise polynomial
interpolation on sparse grids, for instance [12, 44, 46]. In sparse grid kernel
collocation, the sparse grid is decomposed into a number of sub-grids firstly. In
that case, all solutions that are constructed by solving collocation problems on each
sub-grid are linearly combined to form a final solution on the sparse grid. The
details about sparse grid kernel interpolation are discussed completely in [109],
and here we present a particular case to introduce the collocation algorithm.
Suppose u is target function mapping from domain 


\section{Algorithm Details}

\subsection{SiK-c}
The basic Sparse Grid Col-located Interpolation Kernel (SiK-c)

N- matrix

\subsection{MuSik-c}
MuSiKc on the otherhand use multi-level co-located interpolation, where the results of each level are re-used as inputs for the next. As such, MuSiK-c is an inherently sequential evolution of SiKc.\\
 The following diagram shows the processing flow of the algo.\\

\includegraphics[scale=0.3]{MuSiKc1.png}

\subsection{Computational Complexity Comparisons}

 

\section{Implentation Details}

\subsection{Eigen API and MatLab comparison}

\subsection{Eigen Expression Trees vs MatLab}

\subsection{CUDA parallelisation vs Threading}


\section{Appendices}

\subsection{Code Repositories}

\subsection{Supporting Documents}
 

Numerical Experiments

\section{Lagrange Functions in Buhmann Form}
This note contains my view of the now classical Buhmann cardinal
function theory, i.e. the Lagrange function theory, all of which is
a footnote to the Poisson Summation Formula, in the best possible
sense. All of this is probably somewhere in \cite{mdb}, albeit
implicitly or in changed guise. We begin with the univariate theory
for simplicity.

\section{Lagrange Functions on $\ZZ$}

We begin with the classical Poisson Summation Formula in one
dimension, then develop the Buhmann form of the Lagrange function on $\ZZ$.

\subsection{The Poisson Summation Formula on $\ZZ$}

Let $f \in S(\RR)$, to avoid analytical inconvenience.  We need the
classical form of the Poisson Summation Formula. To this end, we
define $\TT := [-\pi, \pi]$ and define the {\em $\TT$-periodization}
of $f$ by
\be
P_{\TT} f(x) \equiv P\!f(x) := \sum_{j \in \ZZ} f(x + 2 \pi j).
\label{PSF0}
\ee

\begin{thm}
\be
P\!f(x) = \sum_{j \in \ZZ} f(x + 2\pi j)
=
(2\pi)^{-1}\sum_{k \in \ZZ} \fhat(k) e^{ikx}.
\label{ps1}
\ee
\label{PSF1}
\end{thm}

\begin{proof}
The smoothness and decay of $f$ imply that the
Fourier series 
\[
P\!f(x) = \sum_{\ell \in \ZZ} c_\ell e^{i\ell x}
\]
converges absolutely and uniformly. Further,
\begin{align*}
c_\ell
&= (2\pi)^{-1} \int_{-\pi}^{\pi} P\!f(x) e^{-i\ell x}\,dx\\
&= (2\pi)^{-1} \int_{-\pi}^{\pi} \left(\sum_{j \in \ZZ} f(x + 2\pi j)\right)
e^{-i\ell x}\,dx\\
&= (2\pi)^{-1} \int_\RR f(x) e^{-i\ell x}\,dx\\
&= (2\pi)^{-1} \fhat(\ell),
\end{align*}
using the Dominated Convergence Theorem to justify the interchange of
summation and integration.
\end{proof}

If we replace $f$ by $\fhat$ in Theorem \ref{PSF1}, then we obtain a
{\em dual} Poisson Summation Formula, as it were.

\begin{cor}
We have
\be
\sum_{k \in \ZZ} \fhat(\xi + 2\pi k) = \sum_{j \in \ZZ} f(j)
e^{-ij\xi}.
\label{ps2}
\ee
\label{PSF2}
\end{cor}

\begin{proof}
Replace $f$ by $\fhat$ in Theorem \ref{PSF1}, recalling that 
$\widehat{\widehat{f}}(x) = 2\pi f(-x)$.
\end{proof}

\subsection{The Lagrange Function on $\ZZ$}

Let $\phi \in S(\RR)$, to avoid all analytic inconvenience.
We also want to
choose $\phi$ with interpolation in mind, so we shall also assume that
its Fourier transform $\phihat$ is strictly positive, which implies that
$f$ is a strictly positive definite function.

We want to construct a function $L \in \Span_{k \in \ZZ} \phi( \cdot -
k)$ for which $L(j) = \delta_{oj}$, for $j \in \ZZ$. Such a function
will be called the Lagrange function, by analogy with the Lagrange
form of the interpolating polynomial. Thus, proceeding formally for the moment,
we have
\be
L(x) = \sum_{k \in \ZZ} \lambda_k \phi(x-k)
\label{L0}
\ee
or, in the Fourier domain,
\be
\Lhat(\xi) = \left(\sum_{k \in \ZZ} \lambda_k e^{-ik\xi}\right)
\phihat(\xi) 
\label{L1}
\ee
It is not obvious that \eqref{L1} is well defined, but we shall soon
show that all is well. 
We periodize both sides to form a $2\pi$--periodic function and,
using the Poisson Summation Formula in the form of Corollary
\ref{PSF2}, we find
We obtain
\be
1 \equiv
\sum_{\ell \in \ZZ} L(\ell) e^{-i\ell \xi}
= \sum_{j\in \ZZ} \Lhat(\xi + 2\pi j)
= \left(\sum_{k \in \ZZ} \lambda_k e^{-ik\xi}\right) \left(\sum_{j \in \ZZ}
\phihat(\xi + 2\pi j)\right).
\label{L2}
\ee
Hence, recalling that $\phihat(\xi) > 0$, for all $\xi \in \RR$,
\eqref{L2} implies
\be
\sum_{k \in \ZZ} \lambda_k \exp(-ik\xi) = \frac{1}{\sum_{j \in \ZZ}
\phihat(\xi + 2\pi j)}.
\label{L2b}
\ee
Substituting \eqref{L2b} in \eqref{L1}, we
obtain the Buhmann form of the Fourier transform of the Lagrange
function, that is,
\be
\Lhat(\xi) = \frac{\phihat(\xi)}{\sum_{j \in \ZZ} \phihat(\xi + 2\pi
  j)}.
\ee



\section{Lagrange Functions on $h\ZZ$}

We follow the same route as before.

\subsection{The Poisson Summation Formula on $h\ZZ$}

We could deduce the scaled version of the Poisson Summation Formula 
directly from Theorem \ref{PSF1}, but I prefer
to begin {\em ab initio}. 
We shall now periodize $f$ over $h^{-1}\TT$, for $h > 0$, i.e.
we define
\be
P_{h^{-1} \TT} f(x) = \sum_{j \in \ZZ} f(x + 2\pi h^{-1} j).
\label{PSFh0}
\ee
The {\em scaled
  exponentials} are the $2\pi h^{-1}$-periodic functions defined by
\be
e^h_j(x) := e^{i h jx}, \qquad j \in \ZZ,
\ee
and form a complete orthonormal set with respect to the inner product
\be
\langle
F, G\rangle
= 
\frac{1}{2\pi h^{-1}} \int_{-\pi h^{-1}}^{\pi h^{-1}}
F(s) G(s)^*\,ds.
\ee
In other words, $\{e^h_j : j \in \ZZ\}$ forms a complete orthonormal
set for $L^2(h^{-1}\TT)$ endowed with the normalized inner product
\[
\langle F, G \rangle = \frac{1}{\Vol_1 h^{-1}\TT} \int_{h^{-1}\TT} F G^*.
\]

\begin{thm} We have
\be
P_{h^{-1}\TT} f(x)
= \sum_{j \in \ZZ} f(x + 2\pi h^{-1} j)
=
\left(2 \pi h^{-1}\right)^{-1} \sum_{k \in \ZZ} \fhat(kh) e^{ihkx},
\label{psh1}
\ee
i.e.
\be
P_{h^{-1} \TT} f = \left(\Vol_1 h^{-1}\TT\right)^{-1}
\sum_{k \in \ZZ} \fhat(kh) e^{h^{-1}}_k.
\label{psh1.1}
\ee
\label{PSFh1}
\end{thm}

\begin{proof}
As in Theorem \ref{PSF1}, the Fourier
series
\[
P_{h^{-1} \TT} f(x) = 
\left(\Vol_1 h^{-1}\TT\right)^{-1} \sum_{\ell \in \ZZ} c^h_\ell e^{ih \ell x}
\]
converges absolutely and uniformly, and
\begin{align*}
c^h_\ell
&= \left(2\pi h^{-1}\right)^{-1}
\int_{-\pi/h}^{\pi/h} P_h f(x) e^{-ih \ell x}\,dx\\
&= \left(2\pi h^{-1}\right)^{-1}
   \int_{-\pi h^{-1}}^{\pi h^{-1}} \left(\sum_{j \in \ZZ} f(x + 2\pi h^{-1} j)\right)
e^{-i h \ell x}\,dx\\
&= \left(2\pi h^{-1}\right)^{-1}\int_\RR f(x) e^{-i h\ell x}\,dx\\
&= \left(2\pi h^{-1}\right)^{-1}\fhat(h \ell),
\end{align*}
using the Dominated Convergence Theorem to justify the interchange of
summation and integration.
\end{proof}

The analogous form of Corollary \ref{PSF2} is now fairly clear.

\begin{cor}
\be
\sum_{k \in \ZZ} \fhat(\xi + 2 \pi h^{-1} k)
= h \sum_{\ell \in \ZZ} f(h \ell) e^{-i h \ell \xi}.
\label{PSFh2}
\ee
\end{cor}


\subsection{The Lagrange Function on $h\ZZ$}

We now consider the Lagrange function when interpolating on the
scaled integer grid $h\ZZ$, for $h > 0$.
Thus we define
\be
\phi_h(x) := \phi(h^{-1}x)
\label{Lh00}
\ee
and we now want to construct a function 
$L^h \in \Span_{k \in \ZZ} \phi_h( \cdot -
kh)$ for which $L^h(jh) = \delta_{oj}$, for $j \in \ZZ$. 
It is almost obvious that $L^h (x) = L(h^{-1} x)$, but the
Fourier analysis is satisfying.
Thus we consider
\be
L^h(x) = \sum_{k \in \ZZ} \lambda^h_k \phi_h(x-kh)
\label{Lh0}
\ee
or, in the Fourier domain,
\be
\Lhhat(\xi) = \left(\sum_{k \in \ZZ} \lambda^h_k e^{-ih k\xi}\right)
\phihhat(\xi) 
\label{Lh1}
\ee
We must now periodize \eqref{Lh1} to obtain a $2\pi h^{-1}$-periodic
function, using the scaled Poisson Summation Formula, i.e.
\[
\sum_{k \in \ZZ} \Lhhat(\xi + 2\pi h^{-1} k)
= h \sum_{\ell \in \ZZ} L^h(h \ell)e^{-ih \ell \xi} \equiv h.
\]
Hence \eqref{Lh1} becomes
\be
1 \equiv h^{-1} \sum_{k \in \ZZ} \Lhhat(\xi + 2\pi h^{-1} k)
= h^{-1}\left( \sum_{k \in \ZZ} \lambda^h_k e^{-ih k\xi}\right)
\sum_{m \in \ZZ} \phihhat(\xi + 2\pi h^{-1} m)
\label{Lh2}
\ee
Eliminating $\sum \lambda^h_k \exp(-ih k\xi)$ from \eqref{Lh1} and
\eqref{Lh2}, we obtain
\be
\Lhhat(\xi) = 
\frac{\phihhat(\xi)}{h^{-1} \sum_{m \in \ZZ} 
   \phihhat(\xi + 2\pi h^{-1} m)}
=
h \left(
\frac{\phihat(h\xi)}{\sum_{m \in \ZZ} \phihat(h\xi + 2\pi m)}\right)
= h \Lhat(h\xi).
\label{Lh3}
\ee
Hence $L^h (x) = L(h^{-1} x)$, as expected.


\section{Lagrange functions on $A \ZZ^d$ for $A \in \GL(\RR^d)$}

Let $A \in \GL(\RR^d)$ and $f \in S(\RR^d)$.
shall be using the normalized inner product on $L^2(A^{-1}\TT^d)$, that
is,
\be
\langle F, G \rangle
= \frac{1}{\Vol_d A^{-1}\TT^d} \int_{A^{-1}\TT^d} F(x) G(x)^*\,dx
\ee
and $\Vol_d A^{-1} \TT^d = (2\pi)^d |A|^{-1}$.
The $A^{-1}\TT^d$-periodic
exponentials providing our complete orthonormal sequence are given
by
\be
e^A_k(x) = e^{i\langle k, A x\rangle}, \qquad k \in \ZZ^d.
\ee

\subsection{The Poisson Summation Formula on $A \ZZ^d$}

We define the $A^{-1}\TT^d$-periodization $P_{A^{-1}\TT} f$ by
\be
P_{A^{-1} \TT^d} f(x) = \sum_{j \in \ZZ^2} f(x + 2\pi A^{-1} j).
\label{ml0}
\ee

\begin{thm}
\be
P_{A^{-1} \TT^d} f(x)
= \left(\Vol_d A^{-1} \TT^d\right)^{-1}
\sum_{\ell \in \ZZ^d} \fhat(A^T \ell) e^A_{ell}(xi).
\label{ml1}
\ee
\label{mlthm1}
\end{thm}

\begin{proof}
We have the Fourier series
\[
P_{A^{-1}\TT^d} f(x) = \sum_{k \in \ZZ^d} c^{A}_k e^A_k(x),
\]
where
\begin{align*}
c^{A}_k
&= \frac{1}{\Vol_d A^{-1} \TT^d} 
\int_{A^{-1}\TT^d} 
P_{A^{-1} \TT^d} f(x) e^{A}_{-k}(x)\,dx\\
&= \frac{1}{\Vol_d A^{-1} \TT^d} 
\int_{A^{-1}\TT^d} \int_{\RR^d} f(x) e^{A}_{-k}(x)\,dx\\
&= \frac{1}{\Vol_d A^{-1} \TT^d} 
\int_{A^{-1}\TT^d} \int_{\RR^d} f(x) e^{-i\langle A^T k, x\rangle}\,dx\\
&= \frac{\fhat(A^T k)}{\Vol_d A^{-1} \TT^d} 
\int_{A^{-1}\TT^d} 
\end{align*}
\end{proof}

The dual form takes a similar form.

\begin{cor}
\be
\sum_{k \in \ZZ^d} \fhat(\xi + 2\pi A^{-T} k)
= |A| \sum_{\ell \in \ZZ^d} f(A\ell) e^{-i\langle \ell, A^T \xi\rangle}.
\label{ml2}
\ee
\label{mlcor2}
\end{cor}

\subsection{The Lagrange Function on $A\ZZ$}

We define $\phi_A(x) = \phi (A^{-1}x)$, for $x \in \RR^d$. Then
$\phiAhat(\xi) = |A| \phihat(A^T \xi)$, and it is again almost obvious
that $L^A(x) = L(A^{-1}x)$, so that $\LAhat(\xi) = |A| \Lhat(A^T
\xi)$. For completeness, we again provide the full Fourier derivation.

\begin{thm}
The Fourier transform of the Lagrange function $L^A$ is given by
\be
\LAhat(\xi)
= \frac{\phiAhat(\xi)}{|A|^{-1} \sum_{k \in \ZZ^d} \phiAhat (\xi + 2\pi
  A^{-T} k)}.
\ee
\end{thm}

\begin{thebibliography}{99}
\bibitem{mski2} Georgoulis, E.h., Levesley, J., Subhan, F. (2013), {\em Multilevel sparse kernel-based interpolation}, SIAM Journal on Scientific Computing, 2013, 35(2):A815-A831.
\bibitem{mski1} Subhan, F. (2011), {\em Multilevel sparse kernel-based interpolation}, Ph.D. Thesis, University of Leicester, 2011.
\bibitem{mski0} Zhao, Yangzhang (2017), {\em Multilevel sparse grid kernels collocation with radial basis functions for elliptic and parabolic problems}, Ph.D. Thesis, University of Leicester, 2017.
\bibitem{rbf0} Myers, D.e.,De Iaco, S.,Posa, D.,De Cesare, L. (2002), {\em Space-time radial basis functions}, Computers and Mathematics with Applications 2002 43(3):539-549
\bibitem{rbf1} E. J. Kansa. (1990), {\em Multiquadrics - a scattered data approximation scheme
with applications to computational fluid-dynamics - I.}, Computers and Mathematics with Applications, 19(8-9):127–145, 1990.
\bibitem{rbf2} E. J. Kansa. (1990), {\em Multiquadrics - a scattered data approximation scheme
with applications to computational fluid-dynamics - II.}, Computers and Mathematics with Applications, 19(8-9):147–161, 1990.
\bibitem{rbf3} R. Franke. (1982), {\em Scattered Data Interpolation: Tests of Some Method}, Mathematics of Computation. 38(157):181-200, 1982
\bibitem{rbf4} A. Pena. (2005), {\em Option pricing with radial basis functions: a tutorial.} Technical
report, Wilmott Magazine, 2005.
\end{thebibliography}



\end{document}





